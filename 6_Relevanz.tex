\section{Praktische Relevanz}

Die Entwicklung jedes Systems setzt die Akzeptanz von den Stakeholdern\footnote{Eine Person oder Gruppe, die Interesse 
für das Projekt haben. Es kann zum Beispiel der Beauftragter oder reine Nutzer sein \cite{refip:HSSI}.} voraus. Um dieses Ziel 
zu erreichen, müssen einige Schritte befolgt werden: Erkennung der Wünsche und der Bedürfnissen der Stakeholder; Hervorhebung 
der Anforderungen des zu entwickelndes Systems; Tests des zu gestalteten Produkts; Implementierung des Systems oder 
der Anwendung; Weiterentwicklung des Systems \cite{refbook:RECR}. Alle diese Schritte ermöglichen die Erzeugung eines 
potenziellen akzeptierten Produkts, das von Kunden genutzt wird und besonders soziale und wirtschaftliche Vorteile bringt.
Diese Vorteile werden im nachhinein beschrieben.


\subsection{Wirtschaftliche Vorteile}

Die Firmen, die in der Lage sind, ein akzeptiertes und sichereres Produkt anzubieten, besitzen Wettbewerbsvorteile, 
die für deren Wachstum und Profit extrem wichtig ist, so dass sie in dem konkurrierenden Markt überleben können. Aus 
diesem Grund, stellen die Investitionen in die Forschung für die Sicherheit bei bargeldloser Bezahlung einen sicheren 
Weg für eine profitable Geschäft dar.


Dadurch, dass der Automat rund um die Uhr funktionieren soll, ist es auch für die Lieferanten einfacher einen Termin 
für die Auslieferung der Waren zu finden. Denn es ist im Prinzip gleichgültig, wann die Ware eintrifft. Die Hauptsache 
ist, dass die Waren im \acrshort{cba} möglichst immer verfügbar sind. Das vermeidet auf jedem Fall die Entstehung von 
Stau und indirekt ermöglicht einen besseren Planungen  für die Lieferung, sodass es weniger Treibstoff verwendet wird.


\subsection{Soziale Vorteile}

Diese geplante wissenschaftliche Arbeit soll vor allem Menschen nutzen, die in eher abgelegenen Orten wohnen und somit 
meistens von großen Supermarktketten vernachlässigt werden. Diese Leute müssen meistens langen fahren, um sich mit
Waren des täglichen Bedarfs zu bewahren. Das bringt außer finanziellen Nachteile auch ökologische, da mehr Autos fahren 
müssen und deswegen mehr Abgase aufgelöst werden. Da der Automat 24 Stunden und 7 Tage in der Woche offen sein soll,
können diese potenziellen Kunden auf solche lange Fahrt verzichten und sogar auch mitten in der Nacht oder am Sonntag
ihre Einkauf am \acrshort{cba} tätigen, ohne für die überteuerten Preise von der Tankstelle bezahlen zu müssen. 

Der eigentlich wichtigste Punkt ist die Sicherheit beim Bezahlvorgang, bei der es in der geplanten wissenschaftlichen 
Arbeit hauptsächlich geht. Das Ziel von der Entwicklung eines sicheren Zahlungsverfahren ist, dass die Kunden sich
also keine Sorgen um die Sicherheit bei bargeldloser Bezahlung machen müssen, da die Methoden für die Bezahlung 
auf dem neusten technischen Stand sind. Die Akzeptanz, die durch ein solches Verfahren entsteht, kann auch dazu beitragen,
dass Digital Non-Natives als potenzielle zu Kunden werden.

\subsection{Fazit}

Heutzutage findet die Kriminalität immer mehr auf der digitalen Ebenen statt. Deshalb repräsentiert die Sicherheit bei 
Bezahlung ein wichtiges Mechanismus zum Schutz von Nutzer. In einem isolierten Ort, wie ein Campingplatz, wo mehr 
Leute ihren Urlaub verbringen, sollte dieses Kümmern nicht existieren. Deshalb ist die Entwicklung eines sicheren
bargeldlosen Zahlungsverfahren für einen \acrfull{cba}, der in der Nähe eines Campingplatz installiert werden soll,
der richtige Weg für die Entstehung einer funktionaler und relevanter Technologie für die gesamte Gesellschaft.
