\section{Praktische Relevanz}

Die Entwicklung jedes System setzt die Akzeptanz von Stakeholder\footnote{Eine Person oder Gruppe, die Interesse für das Projekt 
haben oder auch die Beauftragter \cite{refip:HSSI}.} voraus. Um dieses Ziel zu erreichen, müssen einige Schritte gefolgt
werden: Erkennung der Wünsche und der Bedarf des Stakeholders; Hervorhebung der Anforderungen des zu entwicklendes Systems;
Tests des gestalteten Produkts; Implementierung des Systems oder der Anwendung und Weiterentwicklung der
gesamten Konzipierung \cite{refbook:RECR}. Alle diese Schritte ermöglichen die Erzeugung eines potenziellen akzeptierten 
Systems, das weiterverbreitet werden kann.

\textbf{Angepasst - Anfang}

Diese geplante wissenschaftliche Arbeit soll sowohl eine meistens isolierten Region als auch deren Anwohner Vorteile bringen.
Die erste ist normalerweise von großen Supermarktkette vernachlässigt, was die zweite dazu zwingen, reisen zu müssen, um ihre
Einkäufe zu erledigen. Da der Automat 24 Stunden und 7 Tage in der Woche offen sein soll, kann potenzielle der Kunde in \acrfull{cba}
auch mitten in der Nacht oder am Sonntag kaufen, ohne die überteuerten Preise von der Tankstelle bezahlen zu müssen. 

\textbf{Angepasst - Ende}

Diese geplante wissenschaftliche Arbeit soll einen möglichst großen Nutzen haben und zudem sollen auch viele von dem geplanten 
Projekt profitieren. Zum einen hat der \acrfull{cba} an sich schon einen riesen Nutzen, denn dieser wertet eine Region, in der
dieser Automat steht extrem auf. Dieser Autoamt soll 24 Stunden und 7 Tage in der Woche offen sein. Das heißt, auch wenn beim
normalen Einkauf im Supermarkt ein Artikel vergessen wurde, kann der Kunde diesen inm \acrfull{cba} auch mitten in der Nacht 
oder am Sonntag kaufen, ohne die überteuerten Preise von der Tankstelle bezahlen zu müssen. 

Dadurch, dass der Automat rund um die Uhr funktioniert, ist es auch für die Lieferanten einfacher einen Termin für die Auslieferung 
der Waren zu finden. Denn es ist im Prinzip egal, wann die Ware eintrifft. Die Hauptsache ist, dass die Waren im \acrfull{cba}
möglichst immer verfügbar sind. 

\textbf{Angepasst - Anfang}


Der eigentlich wichtigste Punkt ist die Sicherheit beim Bezahlvorgang, bei der es in der geplanten wissenschaftlichen Arbeit 
hauptsächlich geht. Das Ziel von der Entwicklung eines sicheren Zahlungsverfahren ist, dass die Kunden sich also keine Sorgen 
um die Sicherheit bei bargeldloser Bezahlung machen müssen, da die Methoden für die Bezahlung auf dem neusten technischen Stand sind.

Die Akzeptanz, die durch ein solches Verfahren entsteht, kann sogar auch dazu beitragen, Digital Non-Natives als potenzieller
Kunde anzuziehen. Außerdem kann dieses Projekt andere Firmem dazu bringen, mehr in der Forschung für die Sicherheit bei
bargeldloser Bezahlung zu investieren.



\textbf{Angepasst - Ende}


Aus unserer Sicht wäre es natürlich toll wenn wir durch den \acrfull{cba} auch Digital Non-Natives an das Thema der 
bargeldlosen Zahlungsmethoden ranzuführen und somit alle Bezahlvorgänge zu vereinfachen.
Außerdem werden vielleicht einige Firmen auf das Projekt aufmerksam und stecken noch mehr Zeit und Geld in die Forschung
für die Sicherheit bei bargeldloser Bezahlung.


\textcolor{red}{\textbf{Ich denke wir brauchen mehr hier, aber ich habe keine Ahnung. Und irgendwo muss diese Referenz
von Frau Heinemann da sein.}}

