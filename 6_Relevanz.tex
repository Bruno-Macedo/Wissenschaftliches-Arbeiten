\section{Praktische Relevanz}

Die Entwicklung jedes Systems setzt die Akzeptanz von den Stakeholdern\footnote{Eine Person oder Gruppe, die Interesse 
für das Projekt haben. Es kann zum Beispiel der Auftraggeber oder ein reiner Nutzer sein \cite{refip:HSSI}.}. Um dieses Ziel 
zu erreichen, müssen einige Schritte befolgt werden: Erkennung der Wünsche und der Bedürfnissen der Stakeholder; Hervorhebung 
der Anforderungen des zu entwickelndes Systems; Tests des zu gestalteten Produkts; Implementierung des Systems oder 
der Anwendung; Weiterentwicklung des Systems \cite{refbook:RECR}. All diese Schritte ermöglichen die Erzeugung eines 
potenziell akzeptierten Produkts, das von Kunden genutzt wird und zudem soziale und wirtschaftliche Vorteile bringt.
Diese Vorteile werden im folgenden beschrieben.


\subsection{Wirtschaftliche Vorteile}

Die Firmen, die in der Lage sind, ein akzeptiertes und sichereres Produkt anzubieten, besitzen Wettbewerbsvorteile, 
die für deren Wachstum und Gewinn extrem wichtig sind, sodass sie in dem konkurrierenden Markt überleben können. Aus 
diesem Grund, stellen die Investitionen in die Forschung für die Sicherheit bei bargeldloser Bezahlung einen sicheren 
Weg für ein profitables Geschäft dar.


Dadurch, dass der Automat rund um die Uhr funktionieren soll, ist es auch für die Lieferanten einfacher einen Termin 
für die Auslieferung der Waren zu finden. Denn es ist im Prinzip gleichgültig, wann die Ware eintrifft. Die Hauptsache 
ist, dass die Waren im \acrshort{cba} möglichst immer verfügbar sind. 


\subsection{Soziale Vorteile}

Diese geplante wissenschaftliche Arbeit soll vor allem Menschen nutzen, die in eher abgelegenen Orten wohnen und somit 
meistens von großen Supermarktketten vernachlässigt werden. Diese Leute müssen meistens lange fahren, um sich mit
Waren des täglichen Bedarfs einzudecken. Das bringt außer der finanziellen Nachteile auch ökologische, da diese Menschen 
weitere Strecken mit dem Autos fahren müssen und deswegen mehr Abgase entstehen. Da der Automat 24 Stunden und 7 Tage 
in der Woche offen sein soll, können diese potenziellen Kunden auf diese langen Fahrten verzichten und auch mitten in 
der Nacht oder am Sonntag in der Nähe ihres Wohnortes ihren Einkauf am \acrshort{cba} tätigen, ohne für die überteuerten
Preise von der Tankstelle bezahlen zu müssen. 

Der eigentlich wichtigste Punkt ist die Sicherheit beim Bezahlvorgang, bei der es in der geplanten wissenschaftlichen 
Arbeit hauptsächlich geht. Das Ziel von der Entwicklung eines sicheren Zahlungsverfahren ist, dass die Kunden sich
also keine Sorgen um die Sicherheit bei bargeldloser Bezahlung machen müssen, da die Methoden für die Bezahlung 
auf dem neusten technischen Stand sind. Die Akzeptanz, die durch ein solches Verfahren entsteht, kann auch dazu beitragen,
dass Digital Non-Natives potenzielle Kunden werden.

\subsection{Fazit}

Heutzutage findet die Kriminalität immer öfter auf digitaler Ebene statt. Deshalb trägt die Sicherheit bei 
bargeldloser Bezahlung einen großen Teil dazu bei, den Kunden vor solchen Angriffen zu schützen. In einem isolierten Ort,
wie der Campingplatz, bei dem die meisten Leute ihren Urlaub verbringen, sollten diese Sorgen nicht entstehen. Deshalb
ist die Entwicklung eines sicheren bargeldlosen Zahlungsverfahren für einen \acrfull{cba}, der in der Nähe eines 
Campingplatz installiert werden soll, der richtige Weg für den Aufbau einer funktionalen und relevanten Technologie
für die gesamte Gesellschaft.