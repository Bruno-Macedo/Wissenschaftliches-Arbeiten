\section{Praktische Relevanz}

Überzeugen, dass unsere Forschung praktische Relevanz hat. Diese Forschung hat Wirkung außerhalb der
Welt der Informatik hat.

Mit der Implementierung von \acrshort{nfc} und von Smartcards in einer \acrlong{cba} auf einem Campingplatz Umgebung können
wir folgende folgende gewährleisten:

\textcolor{red}{Hier geben wir eine Liste von Vorteile dieses Einsatzes.}


Mit der erfolgreichen Implementierung des xxxxxxxx können wir folgenden Ziele innerhalb eines Unternehmens erreichen:
Meine Liste PUNKT:
\begin{itemize}
    \item Punkt 1
    \item Punkt 2
    \item Punkt 3
    \item Punkt 4
\end{itemize}


Diese geplante wissenschaftliche Arbeit soll einen möglichst großen Nutzen haben und zudem sollen auch viele von dem geplanten 
Projekt profitieren. Zum einen hat der \acrfull{cba} an sich schon einen riesen Nutzen, denn dieser wertet eine Region, in der
dieser Automat steht extrem auf. Dieser Autoamt soll 24 Stunden und 7 Tage in der Woche offen sein. Das heißt, auch wenn beim
normalen Einkauf im Supermarkt ein Artikel vergessen wurde, kann der Kunde diesen inm \acrfull{cba} auch mitten in der Nacht 
oder am Sonntag kaufen, ohne die überteuerten Preise von der Tankstelle bezahlen zu müssen. 

Dadurch, dass der Automat rund um die Uhr funktioniert, ist es auch für die Lieferanten einfacher einen Termin für die Auslieferung 
der Waren zu finden. Denn es ist im Prinzip egal, wann die Ware eintrifft. Die Hauptsache ist, dass die Waren im \acrfull{cba}
möglichst immer verfügbar sind. 

\textbf{Angepasst - Anfang}

Der eigentlich wichtigste Punkt ist die Sicherheit beim Bezahlvorgang, bei der es in der geplanten wissenschaftlichen Arbeit 
hauptsächlich geht. Die Kunden müssen sich also keine Sorgen um die Sicherheit bei bargeldloser Bezahlung machen, da die Methoden 
für die Bezahlung auf dem neusten technischen Stand sind. Die Akzeptanz, die durch ein sicheres Zahlungsverfahren entstehen kann, 
kann sogar Digital Non-Natives als potenzieller Kunde anziehen. Außerdem kann dieses Projekt andere Firmem dazu bringen, mehr
in der Forschung für die Sicherheit bei bargeldloser Bezahlung.

\textbf{Angepasst - Ende}


Aus unserer Sicht wäre es natürlich toll wenn wir durch den \acrfull{cba} auch Digital Non-Natives an das Thema der 
bargeldlosen Zahlungsmethoden ranzuführen und somit alle Bezahlvorgänge zu vereinfachen.
Außerdem werden vielleicht einige Firmen auf das Projekt aufmerksam und stecken noch mehr Zeit und Geld in die Forschung für die Sicherheit bei 
bargeldloser Bezahlung.


\textcolor{red}{\textbf{Ich denke wir brauchen mehr hier, aber ich habe keine Ahnung. Und irgendwo muss diese Referenz
von Frau Heinemann da sein.}}

