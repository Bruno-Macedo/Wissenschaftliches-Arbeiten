\section{Einführung}



Seit einigen Jahren entscheiden sich immer mehr Menschen Urlaub auf einem Campingplatz 
zu machen. Der Gedanke an Menschenmassen und Fallen für Touristen schreckt die Leute von
den typischen Touristenzielen ab. Zudem ist der Kontakt zu der Natur für viele ein wichtiger
Punkt in einem Urlaub. In den letzten anderthalb Jahren stieg die Anzahl von Campinplatzbesuchern
rasant. Die Corona-Pandemie drängte die Leute dazu, Urlaubsmöglichkeiten zu suchen, bei denen
das Risiko von einer Infektion niedrig sei und wo genug Abstand gehalten werden könne. Da viele 
Hotels und andere Ferieneinrichtungen geschlossen waren, blieb vielen Leuten, besonders Familien,
nichts anderes übrig, als die Ferien etwas anders zu organisieren und gestalten.

Die traditionelle Idee von Campingplätzen, bei der Jugendliche oder Familien weit entfernt von der 
Gesellschaft sind, ist heute eine andere. Heute wollen Urlauber auf den Kontakt mit der Natur
möglichst nicht verzichten, wodurch Campingplätze immer voller werden. Aus diesem Grund wäre es
sinnvoll, die Möglichkeiten zur Grundversorgung zu erweitern, ohne direkt einen neuen Supermarkt
bauen zu müssen. In dieser Hinsicht kann die Einrichtung eines elektronischen Click-and-Buy-
Supermarktes, der mit einem Automaten zu vergleichen ist, eine wesentliche Rolle spielen, um 
einen Campingplatz und ie Gegend drum herum zu modernisieren, die Möglichkleiten zur Grundversorgung zu erweitern und 
ihn attraktiver für Reisende zu machen.

Die Sicherheit eines digitalen Konzepts stellt eines des wichtigsten Punkte für die Entwicklung
eines Systems dar. Vernachlässigungen in diesem Bereich führen zu unberechenbaren Vertrauensverlust 
seitens der potenziellen Nutzenden und zu finanziellen und moralischen Schäden der direkten
Stakeholders. Der folgende Artikel soll beschreiben, welche Schritte auf digitaler Sicherheitsebene eingeleitet 
werden müssen, um einen akzeptablen und funktionierenden Click and Buy Automat errichten zu können. 