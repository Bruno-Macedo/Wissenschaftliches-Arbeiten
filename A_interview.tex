\section{Interview}\label{appendix:interview}

Aus dem Interview mit den zwei recherchierten Firmen zielen wir drauf ab, folgende qualitative Informationen hervorzuheben:

\begin{itemize}
    \item Zusammenfassung der Geschichte der Firma
    \item Hierarchische Struktur
    \item Informationen über das Sicherheitsteam: Spezialitäten und Anwendungsbereich
    \item Anzahl der Mitarbeiter in dem IT-Sicherheitsbereich
    \item Qualifikationen der Mitarbeiter des Sicherheitsbereiches
    \item Voraussichtliches Budget für die Sicherheitsentwicklung
    \item Aktuelle finanzielle Investitionen in dem IT-Sicherheitsbereich
    \item Messung der Akzeptanz der ausgelieferten Produkten
\end{itemize}

Die unteren Frage sollen als Orientierung dienen. Während des Gespräches können andere Themen oder Fragen zusätzlich
auftauchen:

\begin{enumerate}
    \item Wie lange existiert ihre Firma schon und seit wann beschäftigen Sie sich intensiv mit dem Bereich IT-Sicherheit?
    \item Wie viele Mitarbeiter gibt es insgesamt in der Firma?
    \item Wie viele Mitarbeiter hat der Bereich, der für die IT-Sicherheit zuständig ist?
    \item Ist ein großes Budget notwendig, um intensiv und erfolgreich in dem Bereich der IT-Sicherheit zu forschen?
    \item Was könnten wichtige Punkte sein, die wir bei unserer Forschung in Bezug auf IT-Sicherheit beachten sollen?
    \item Ist der Bereich sehr speziell, dass man nach dem Studium/Ausbildung noch eine Weiterbildung braucht oder 
    reicht einfach eine Ausbildung/Studium?
    \item Nennen Sie mir Punkte, die in der IT-Sicherheit ihrer Meinung nach besonders wichtig sind.
    \item Welche Produkte haben sie für den Bereich der IT-Sicherheit schon entwickelt?
    \item Was ist der größte Kostenpunkt in der IT-Sicherheit?
    \item Für wie wichtig schätzen Sie den IT-Sicherheitsbereich für die Zukunft ein?
\end{enumerate}
