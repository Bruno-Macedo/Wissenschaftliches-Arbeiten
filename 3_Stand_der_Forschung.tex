\section{Stand der Forschung}


Die zunehmende Tendenz in Deutschland von bargeldloser Bezahlung erfordert neuen Umgang mit den 
eigegebenen Daten. Eine Studie von 2009 der Deutschen Bundesbank zeigte den rasanten Anstieg von 
bargeldloser Bezahlung in der Bundesrepublik seit der Einführung von solchen Zahlungsmethoden 
\cite{refrep:DBCP}.

\begin{figure}[htb]
    \centering{\includegraphics[width=5cm]{Bilder/refrep_DB.png}}
    \caption{Cashless payments via the Deutsche Bundesbank\\ (Bundesbank, 2009, S.52)}
    \label{fig:refrep_DB}
\end{figure}


Laut einer Statistik des Handelsforschungsinstituts EHI von 2019 \cite{refart:KSDL} bezahlen 48,6\% 
der deutschen ihre Waren mit Karte, wohingegen nur noch 46,9\% der deutschen den klassischen 
Weg mit Bargeld gehen. Auch das kontaklose Bezahlen, bei dem kleine Beträge nicht einmal mit einer 
PIN bestätigt werden müssen, nimmt immer weiter zu. Doch gerade bei dieser Variante ist es sehr einfach
im Namen eines anderen zu bezahlen, was eine Sicherheitsrisiko darstellt.


Immer wenn mit Karte bezahlt wird, gehen die Kunden davon aus, dass die Zahlungsabwicklung sicher ist. 
Wie sicher ist das bargeldlose Zahlen heutzutage wirklich? 


Aus diesem Grund ist Vertraulichkeit die erste und wichtigste Voraussetzung, dass ein solches System 
erfüllen muss, um neue potenzielle Kunden zu gewinnen. Unter diesem Begriff soll ein System nur auf 
autorisierte Informationen zugreifen \cite{refbook:SWIS}. In dieser Hinsicht ist die Entwicklung 
einer Click and Buy Maschine so zu konzipieren, dass sie einen sicheren Umgang mit den Kundendaten
anbietet. Diese Interaktion zwischen Kunde und systemkritischen Mechanismen wurde von 
\cite{refart:HARE} so dargestellt:

\vfill
\begin{figure}[htb]
    \centering{\includegraphics[width=5cm]{Bilder/refark_HARE.png}}
    \caption{Sicherheitseigenschaften von digitalen Zahlungsmethode (Hassan et al. 2020, S8)}
    \label{fig:refark_HARE}
\end{figure}
\vfill

Außerdem sollen die anderen Schutzziele der IT-Sicherheit: Integrität, Verfügbarkeit und
Authentizität auch berücksichtigt werden, sodass die Systemen einwandfrei und sicher funktionieren.
Eine Zahlungsmethode, bei der alle Vorraussetzungen erfüllt werden, kann in der Lage sein, das Vertrauen und 
die Akzeptanz von den Nutzenden zu bekommen \cite{refart:HARE}. 


\cite{inbook:MHNS} nennt solche Maschinen Cyber-Physical System (CPS), weil sie eine Interaktion zwischen 
Nutzer und einem oder vielen Systemen darstellt. In dieser Zusammenarbeit spielt der Datenaustausch 
eine wesentliche Rolle, besonders von der Seite der Nutzenden. Diese Technologie zielt eine günstigere 
Entwicklung, ohne die Sicherheit zu vernachlässigen. Diese Interaktion findet erfolgereich statt, 
wenn die genannten Sicherheitszielle erfüllt werden.


Da es um einen dynamischen Sektor geht, bei dem es sehr schnell zu Änderungen kommen kann, \cite{refip:NYRS} 
muss die Technologie stets weiterentwickelt und angepasst werden, um Vertraulichkeitsverlust seitens
der Kunden zu vermeiden. Da die Vertraulichkeit noch nicht zu 100 Prozent gewähleistet werden kann,
verweigern viele Kunden das bargeldlose Bezahlen.

\textbf{\textcolor{red}{Ab hier können wir dann versuchen, die Informatien aus den Artikeln zu nehmen, solche
die du hier hinzugefügt hast und solche, die ich dir am Fr schickte}}

\subsection{Drahtlose Verbindungen und Sicherheit bei Bezahlungen}

Viele digitale Zahlungen finden über WLAN statt, das kann eine großere Risiko darstellen \cite{refip:NYRS}, 
da WLAN-Verbindungen nicht so sicher sind wie Kabelverbindungen. Maßnahmen zu entwickeln, die sich an 
verschiedenen Systemen anpassen, kosten Zeit und Investitionen von Banken und Sicherheitsfirmen. Für jeden 
möglichen Angriffe sollte präventiv etwas getan werden, sodass die Integrität des Kunden
geschützt bleibt. Die folgenden Schwachstellen bei digitaler Zahlung wurden von \cite{refip:NYRS} 
zusammengefasst:

\begin{itemize}
    \item Erstellung von Dateien in dem Opfersystem mit umfangreichen Privilegien;
    \item Unzureichende Sicherheit bei der Validierung von Zertifikaten;
    \item Quellcode ist öffentlich zugänglich, sodass das Opfersystem von Reverse Engineering betroffen sein könnte
    \item Unsicherer Umgang mit Cookies-Einstellungen
\end{itemize}

\cite{refip:NYRS} schlägt einige Sicherheitsmechanismen vor, die die oben gennanten Schwachstellen bei 
kabelosen Verbindungen reduzieren können. Unter denen werden folgende hervorgehoben: 

\begin{itemize}
    \item Nutzung von modernen kryptographischen Standards für die Validierung von Zertifikaten;
    \item Erstellung von Loggdatei, sodass jeder Anormalität schnell überprüft werden kann;
    \item Zwei-Faktor-Authentifizierung;
    \item Digitale und zufällig geordnete Tastatur;
    \item Schwierigkeitsgrad bei der Erstellung von Passwörtern;
    \item Besserer Umgang mit der Verwaltung von Cookies;
    \item Registrierung von Geräten;
    \item Künstliche Intelligenz (KI) für die Detektion von abweichenden Verhalten;
    \item Ständige Kontrolle gegen Social Engineering.
\end{itemize}

Da drahtlose Zahlungen bei Campingplätzen eine wesentliche Rolle spielen kann, muss die Sicherheit solcher
Zahlungsart gewährleistet werden. Das kann erfolgreich passieren, wenn Banken und andere Finanz
Institutionen sich intensiv mit den verschiedenen Angriffsmöglichkeiten und deren Schutzmaßnahmen beschäftigen.

Zahlungskarten wie Kredit- oder EC-Karte sollen auch Zahlungsmittel bei einem Click and Buy Maschinen
zur Verfügung gestellt werden. In Bezug auf diese Modalitäten werden die verschiedenen Aspekte der 
Sicherheit dieser Zahlungsart unten beschrieben.


\subsection{Anwendung von Smartcards und sicheres Bezahlen}
Der Begriff Smartcards bezeichnet eine Plastikkarte mit einem eingebauten Chip, der ein eigenes Betriebssystem,
einen Mikroprozessor und minimale Funktionalitäten besitzt \ref{fig:eigenes_Bild}. 

\vfill
\begin{figure}[htb]
   \centering{\includegraphics[width=10cm]{Bilder/eigenes_Bild_Karte.png}}
   \caption{Eine Smartcard und deren eingebetete Mikrochip\\(eigene Quelle)}
   \label{fig:eigenes_Bild}
\end{figure}
\vfill

Sie wurde vor mehr als 40 Jahren erfunden und ihr Ziel ist die Sicherheit von Kartenzahlung und allgemeine
Authentifizierungsverfahren zu erhöhen \cite{refip:JFSB}. Sie unterscheiden sich von traditionelen 
Magnetstreifenkarten, weil sie verschiedene Authentifizierungsmethoden ermöglichen auch ohne direkte 
Verbindung zur Bank \cite{refbook:ATMS}. Im folgenden wird der Authentifizierungsprozess einer Smartcard 
\ref{fig:refbook_ATMS} dargestellt. 


%\vfill
%htb
\begin{figure}[H]
    \centering{\includegraphics[width=10cm]{Bilder/refbook_ATMS.png}}
   \caption{Authentifizierungsprozess von Smartcards\\(Tanenbaum, 2009, S.755)}
   \label{fig:refbook_ATMS}
\end{figure}
%\vfill


Wenn es um die Sicherheit von Smartcards geht, beschreibt \cite{refmas:ASSS}, dass die Angriffe sich auf 
Hardwareebene konzentrieren. Er beschreibt folgende Techniken für Angriffe:

\begin{itemize}
    \item Protokollanalyse: schwache Konzipierung oder mangelnde Verschlüsselung ermöglichen den Zugang 
    zu dem Klartext; 
    \item Relay: Konzentriert auf kontaktlose Smartcards, um den Inhalt umzuleiten;
    \item Seitenkanal-Attacken: zielt nicht direkt den Inhalt des Kommunikation, sondern versucht sie
    irgendwie zu stören;
    \item Hardware Reverse Engineering: Verständnis über die Algorithmen oder Extrahieren des Schlüssels.
\end{itemize}


Die Schutzmaßnahmen können sich laut \cite{refmas:ASSS} in drei Gruppe geteilt werden: physikalisch,
logisch und organisatorisch. Auf der physikalischen Ebene soll der Hardware robust aufgebaut werden,
um Angriffe schwieriger zu machen. Dieser Konstruktion soll komplex mit zusätzlichen Elementen eingebaut
werden und Sensoren können auch dazu beitragen, das System auszuschalten bzw. zu blockieren, falls es zu 
einer verdächtigen Nutzung kommt. Auf der logischen Ebenen sollen moderne und starke Verschlüsselungsmechanismen
eingesetzt werden, dessen Verfahren gegen Sicherheitslücken überprüft wurde. Die Bearbeitungszeit spielt hier
eine wesentliche Rolle gegen Angriffe, die auf Seitkanäle basiert sind. Letztendlich sollen solche Smartcards
auf dem organisatorischen Ebenen in der Lage sein, Angriffe schnell zu detektieren und zu verhindern. Zu dieser
Ebene gehören auch die Sperre der Karte, die Überprüfung von Logdateien, um Klone zu identifizieren, und
die Verwendung von Zwei-Faktor-Authentisierung.



\begin{enumerate}
    \item \textbf{\textcolor{red}{Sicherheitslücken}}
    \item \textbf{\textcolor{red}{Sicherheitsmechanismen}}
\end{enumerate}

\textbf{\textcolor{red}{Ich habe Teil des Artikels sehr schnell gelesen. Ich denke, wir können dessen Inhalt hier
irgendwie zusammenfassen. In diesem Fall sprechen wir dann über verschiedene Sicherheitsrisiken und Gegenmechanismus
zuerst für WLAN dann mit Karte usw. \cite{refmas:ASSS}}}

\textbf{\textcolor{red}{Ich würde so machen:}}



\begin{enumerate}
    \item \textbf{\textcolor{red}{Sicherheitslücken}}
    \item \textbf{\textcolor{red}{Sicherheitsmechanismen}}
\end{enumerate}

\textbf{Was sind Smartcards}
\begin{enumerate}
    \item \textbf{\textcolor{red}{Hier müssen wir recherchieren. Über was sie sind, würde ich in einer anderen QUelle Schutzmaßnahmen}}
    \item \textbf{\textcolor{red}{Wo werden sie bentutzt. }}
    \item \textbf{\textcolor{red}{Authentifizierung: PIN, CHIP, Mehrfachautentifzierung}}
    \item \textbf{\textcolor{red}{Auch ohne Pin für kleinen Beitra}}
    \item \textbf{\textcolor{red}{kurze beschreibung von Angriffetechnick: Protokollanalyse, Relay   Hardware Reverse Engineering, Angriffebeispiel: legic prime, mifare  }}
    \item \textbf{\textcolor{red}{Gegenmassnahmen: physikalisch, logisch: authentifzierung, besserere verschlüssung. Andere Quelle suchen }}
\end{enumerate}


\vfill
\begin{figure}[htb]
    \centering{\includegraphics[width=5cm]{Bilder/refmas_ASSS.png}}
    \caption{Abbildung von Smartcards - Das kann später verbessert werden}
    \label{fig:refmas:ASSS}
\end{figure}
\vfill



\vspace{2cm}
\textbf{Ich würde diesen Satz in den nächsten Kapitel verwenden und erweitern mit unseren Recherchen, damit wird 
die Literatur rechtfertigen können}
Um das zu bewerkstelligen, ist der aktuelle technische Stand von entscheidener Bedeutung. 
Ausgehend von dieser Informationen muss das Glasfasernetz eventuell erweitert oder auch neu verlegt werden.
Denn das Ziel ist es, technisch gesehen auf dem neusten Stand zu sein, damit das Click and Buy System für die Zukunft abgesichert ist.
Außerdem wird durch den Ausbau des Glasfasernetzes die Region insgesamt deutlich attraktiver gemacht, was vielleicht auch Menschen dazu bringt
in diese Region zu ziehen. Denn jedem ist klar, dass ein guter Internetausbau essentiell ist, um vielleicht auch mal von zuhause aus zu arbeiten.