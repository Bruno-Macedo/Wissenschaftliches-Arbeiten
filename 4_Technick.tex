\section{Stand der Technik}

Für die Bezahlungsmethoden werden hier zwei/drei verschiedene Arten in Hinsicht auf ihrer Schwachstellen
und ihrer Hartungen betrachtet: drahtlose Zahlung und Kartenzahlung.

\subsection{Drahtlose Verbindungen und Sicherheit bei Bezahlungen}

Viele digitale Zahlungen finden über WLAN statt, das kann eine großere Risiko darstellen \cite{refip:NYRS}, 
da WLAN-Verbindungen nicht so sicher sind wie Kabelverbindungen. Maßnahmen zu entwickeln, die sich an 
verschiedenen Systemen anpassen, kosten Zeit und Investitionen von Banken und Sicherheitsfirmen. Für jeden 
möglichen Angriffe sollte präventiv etwas getan werden, sodass die Integrität des Kunden
geschützt bleibt. Die folgenden Schwachstellen bei digitaler Zahlung wurden von \cite{refip:NYRS} 
zusammengefasst:

\begin{itemize}
    \item Erstellung von Dateien in dem Opfersystem mit umfangreichen Privilegien;
    \item Unzureichende Sicherheit bei der Validierung von Zertifikaten;
    \item Quellcode ist öffentlich zugänglich, sodass das Opfersystem von Reverse
    Engineering betroffen sein könnte
    \item Unsicherer Umgang mit Cookies-Einstellungen
\end{itemize}

\cite{refip:NYRS} schlägt einige Sicherheitsmechanismen vor, die die oben gennanten Schwachstellen bei 
kabelosen Verbindungen reduzieren können. Unter denen werden folgende hervorgehoben: 

\begin{itemize}
    \item Nutzung von modernen kryptographischen Standards für die Validierung von Zertifikaten;
    \item Erstellung von Loggdatei, sodass jeder Anormalität schnell überprüft werden kann;
    \item Zwei-Faktor-Authentifizierung;
    \item Digitale und zufällig geordnete Tastatur;
    \item Schwierigkeitsgrad bei der Erstellung von Passwörtern;
    \item Besserer Umgang mit der Verwaltung von Cookies;
    \item Registrierung von Geräten;
    \item Künstliche Intelligenz (KI) für die Detektion von abweichenden Verhalten;
    \item Ständige Kontrolle gegen Social Engineering.
\end{itemize}

Da drahtlose Zahlungen bei Campingplätzen eine wesentliche Rolle spielen kann, muss die Sicherheit 
solcher Zahlungsart gewährleistet werden. Das kann erfolgreich passieren, wenn Banken und andere Finanz
Institutionen sich intensiv mit den verschiedenen Angriffsmöglichkeiten und deren Schutzmaßnahmen 
beschäftigen.

Zahlungskarten wie Kredit- oder EC-Karte sollen auch Zahlungsmittel bei einem Click and Buy Maschinen
zur Verfügung gestellt werden. In Bezug auf diese Modalitäten werden die verschiedenen Aspekte der 
Sicherheit dieser Zahlungsart unten beschrieben.


\subsection{Anwendung von Smartcards und sicheres Bezahlen}
Der Begriff Smartcards bezeichnet eine Plastikkarte mit einem eingebauten Chip, der ein eigenes 
Betriebssystem, einen Mikroprozessor und minimale Funktionalitäten besitzt \ref{fig:eigenes_Bild}. 

\vfill
\begin{figure}[htb]
   \centering{\includegraphics[width=10cm]{Bilder/eigenes_Bild_Karte.png}}
   \caption{Eine Smartcard und deren eingebetete Mikrochip\\(eigene Quelle)}
   \label{fig:eigenes_Bild}
\end{figure}
\vfill

Sie wurde vor mehr als 40 Jahren erfunden und ihr Ziel ist die Sicherheit von Kartenzahlung und allgemeine
Authentifizierungsverfahren zu erhöhen \cite{refip:JFSB}. Sie unterscheiden sich von traditionelen 
Magnetstreifenkarten, weil sie verschiedene Authentifizierungsmethoden ermöglichen auch ohne direkte 
Verbindung zur Bank \cite{refbook:ATMS}. Im folgenden wird der Authentifizierungsprozess einer Smartcard 
\ref{fig:refbook_ATMS} dargestellt. 


%\vfill
%htb
\begin{figure}[H]
    \centering{\includegraphics[width=10cm]{Bilder/refbook_ATMS.png}}
   \caption{Authentifizierungsprozess von Smartcards\\(Tanenbaum, 2009, S.755)}
   \label{fig:refbook_ATMS}
\end{figure}
%\vfill


Wenn es um die Sicherheit von Smartcards geht, beschreibt \cite{refmas:ASSS}, dass die Angriffe sich auf 
Hardwareebene konzentrieren. Er beschreibt folgende Techniken für Angriffe:

\begin{itemize}
    \item Protokollanalyse: schwache Konzipierung oder mangelnde Verschlüsselung ermöglichen den Zugang 
    zu dem Klartext; 
    \item Relay: Konzentriert auf kontaktlose Smartcards, um den Inhalt umzuleiten;
    \item Seitenkanal-Attacken: zielt nicht direkt den Inhalt des Kommunikation, sondern versucht sie
    irgendwie zu stören;
    \item Hardware Reverse Engineering: Verständnis über die Algorithmen oder Extrahieren des Schlüssels.
\end{itemize}


Die Schutzmaßnahmen können sich laut \cite{refmas:ASSS} in drei Gruppe geteilt werden: physikalisch,
logisch und organisatorisch. Auf der physikalischen Ebene soll der Hardware robust aufgebaut werden,
um Angriffe schwieriger zu machen. Dieser Konstruktion soll komplex mit zusätzlichen Elementen eingebaut
werden und Sensoren können auch dazu beitragen, das System auszuschalten bzw. zu blockieren, falls es zu 
einer verdächtigen Nutzung kommt. Auf der logischen Ebenen sollen moderne und starke Verschlüsselungsmechanismen
eingesetzt werden, dessen Verfahren gegen Sicherheitslücken überprüft wurde. Die Bearbeitungszeit spielt hier
eine wesentliche Rolle gegen Angriffe, die auf Seitkanäle basiert sind. Letztendlich sollen solche Smartcards
auf dem organisatorischen Ebenen in der Lage sein, Angriffe schnell zu detektieren und zu verhindern. Zu dieser
Ebene gehören auch die Sperre der Karte, die Überprüfung von Logdateien, um Klone zu identifizieren, und
die Verwendung von Zwei-Faktor-Authentisierung.
