\section{Forschungsplan}

Das Thema Netzwerksicherheit beinhaltet viele Forschungsrichtungen, die zu umfangreich für eine einfache
Recherche sind. Aus diesem Grund und aus Knappheit von Platz konzentrieren wir uns in der geplanten 
wissenschaftlichen Arbeit auf zwei spezifische Aspekte dieses Themas, und zwar auf Schwachstellen und 
auf Härtungsmaßnahmen von \acrshort{nfc} und von Smartcards. Die verwendeten Methoden dieser Untersuchung 
sollen sowohl quantitative als auch qualitative Daten hervorheben, die dabei helfen sollen, die Sicherheit
von Zahlungsverfahren, festzulegen und zu implementieren \cite{refbook:RMJL}. Um an vertrauenswürdige 
und wissenschaftliche Informationen für die geplante wissenschaftliche Arbeit zu gelangen, verwenden wir die unten
beschriebenen Methoden:

\begin{itemize}
  \item Interview mit der Firma, die die \acrfull{cba} erstellt
  \item Durchführung von Experimenten mit Smartcards und \acrshort{nfc}
  \item Beobachtung von Angriffsmöglichkeiten
  \item Literaturrecherche
\end{itemize}

Der IT-Bereich entwickelte seine eigenen Forschungsmethode auf Basis von anderen Fachrichtungen \cite{inbook:AHDS}.
Aus diesem Grund müssen sowohl die Recherche als auch ihre Darstellung entsprechend angepasst werden, sodass 
die Forschung selbst und deren Ergebnisse verständlich präsentiert werden können \cite{refbook:RMJL}. Da Forschung und 
ihre Methoden nicht in Stein gemeißelt sind, spielen Flexibilität und Vielfältigkeit der Quellen eine wichtige 
Rolle für die Entwicklung einer erfolgreichen und glaubwürdigen Untersuchung.

Jedes Element der geplanten wissenschaftlichen Arbeit soll so konzipert werden, sodass sie der Richtlinien von
\cite{refip:SGRM} für die Entwicklung von Forschungen im IT-Bereich entsprechen. Die verwendeten Methoden 
sollen eine theoretische und praktische Abbildung des Objekts dieser Untersuchung zeigen, um ihre Anwendung 
direkt in der realen Welt darzustellen. Im folgenden werden die diversen Methoden der geplanten wissenschaftlichen 
Arbeit ausführlich beschrieben. Die folgende Abbildung soll den Rechercheweg der geplanten wissenschaftlichen 
Arbeit verdeutlichen.


\begin{figure}[H]
  \centering{\includegraphics[width=12cm]{Bilder/forschungsdesign.png}}
  \caption{Forschungsdiagramm
  \\ Quelle: eigene Darstellung}
  \label{fig:FD}
\end{figure}


%\begin{landscape}
%  \thispagestyle{mylandscape}
%  \begin{figure}[h]  
%  \centering
%    \centering{\includegraphics[width=20cm]{Bilder/Diagram2.png}}
%         \caption{Recherchespfad\\ Quelle: eigene Darstellung}
%          \label{fig:diagramfrage}
%  \end{figure}
%\end{landscape}

\subsection{Interview mit Click-and-Buy-Automat Firma}

Die Sicherheit eines Bezahlungssystem steht im Mittelpunkt jeder Firma, die \acrfull{cba} entwickeln. 
Für diese Recherche wollen wir Interviews mit IT-Sicherheitsmitarbeiter zwei in Deutschland basierten Firmen, 
die dieses Art von Geschäft schon anbieten: ``REWE digital'' und ``myenso''. In diesem Fall arbeiten wir mit Firmen, 
die zwar ein ähnliches Dienst anbieten, aber verschiedene Einsätze haben. Während die erste ein großes Unternehmen
mit mehr als 300.000 Mitarbeiter ist \cite{refst:REWE}, ist die zweite ein kleines und neues Unternehmen, die weniger
als 100 Mitarbeiter beschäftigt \cite{refst:MYENSO}.

``REWE digital'' gehört der REWE Group und ist dafür zuständig, die Marke zu digitalisieren. Die Firma hat sich
in verschiedene Bereich der Digitalisierung entwickelt, wie Liefer- und Abholservice und Mobile Awnedung. ``myenso'' 
will ein neues Konzept von Einkauf anbieten, indem die Kunden mehr Entscheidung und Teilnahmen bekommen. Im Vergleich 
zu großeren Ketten will ``myenso'' die Kunde in Orten erreichen, wo normalerweise weniger Einkaufsmöglichkeit gibt
oder die nicht zu Interesse der großeren Ketten gehört.


Für die Interviews stellen wir sowohl quantitative als auch qualitative Interviewsfragen. Aus dem quantitativen Fragenkatallog
wollen wir messbare Daten hervorheben \cite{refbook:SRJR}, die der Umgang der Firma mit Sicherheit in Zahlungsverfahren
beschreibt: Anzahl von \acrfull{cba} und von Mitarbeiter, die sich nur mit digitalen Sicherheitsverfahren beschäftigen;
und Beschreibung mögliche Angriffe, die die Maschine Opfer sein können. Aus der qualitativen Fragensammlung sollen Verfahren 
und Mechanismen gefragt werden, die die Firmen verwendet, um das Zahlungsverfahren ihrem Dienst zu gewährleisten. 
Sowohl die Entwicklung von Zahlungsmethode bei den Produkten als auch der aktuelle Stand der Zahlungsverfahren
sollen in dieser Umfrage gedeckt sein. Für die qualitative Datenerhebung wird Methode von Fokusgruppe\footnote{Fokusgrupp
aus dem Englischen ``focus group'' bezeichnet eine art von qualitativen Diskussion, wo die Teilnehmer mithilfe eines 
Moderator ein Thema besprechen \cite{refbook:APGF}.} verwendet, um die wichtigen Anforderungen im Bezug auf Sicherheit von 
Zahlungsverfahren zu finden, zu analysieren und zu bewerten. Die gezielten und auch gleichzeitig offenen Fragen sollen den
Befragter die Möglichkeite anbieten \cite{refbook:EFAF}, über die existierenden Schwachstellen der angebotenen Dienst zu äußern 
und auch über verwendeten oder noch zu verwendeten Sicherheitsmaßnahmen.

\textcolor{red}{\textbf{Die unteren Quellen und deren Themen:}}
\begin{itemize}
  \item \cite{refbook:RECR}: Requirements-Engineering und -Management: 
  Das Handbuch für Anforderungen in jeder Situation \textcolor{red}{\textbf{Dieses Buch MUSS irgendwo
  zitiert werden. Es wird von Frau Heinemann in ihrem Unterricht verwendet }}
  \item \cite{refbook:CESR}: Systematisches Requirements Engineering und Management 
  - Anforderungen ermitteln,   spezifizieren, analysieren und verwalten
  \item \cite{refbook:DMFG}: he Focus Group Guidebook - 
  \textcolor{red}{\textbf{Über dieses Thema habe ich shcon geschrieben.}}
  \item \cite{refbook:TEUF}: Triangulation: Eine Einführung
  \item \cite{refbook:BPAG}: Biografieforschung als Praxis der Triangulation
  \item \cite{refbook:RMJL}: Scientific Research in Information Systems - A Beginner's Guide
\end{itemize}


\subsection{Durchführung von Experimenten}
\textcolor{red}{\textbf{Hier bin ich mir nicht ganz sicher, ob wir schreiben, als ob wir die Experimenten 
schon ausgeführt haben oder wie wir es ausführen wollen.}}

Die Tests für die Objekte dieser Untersuchung sollen im Labor der Hochschule Worms durchgeführt werden. Dazu werden 5 
Maschinen verwendet, die folgende Rollen übernahmen sollen: Server, Host und Angreifer und 2 Leerlauf-Maschine oder 
\textit{Zombie-botnet}\footnote{Leerlaufe, \textit{idle} oder \textit{Zombie-botnet} bezeichnen Maschine, die für Angriff
verwendet werden. In den meisten Fälle sind die Nutzer dieser Maschine nicht bewusst, dass Angreifer ihre Maschine für
diesen Zweck verwenden \cite{refart:XGDD}.}. Der Host soll eine Anfrage an den Server schicken, die eine Simulation
von einem Bezahlvorgang darstellen soll. Der Server soll unter normalen Umstände auf diese Anfrage antworten und unter
einem Angriff keine Antwort geben. Dieses Verfahren findet sowohl bei Drahtlosen Verbindungen als auch bei Smartcards statt.


\subsubsection{Angriff und Härtungsmaßnahme einer drahtlosen Server}
Für dieses Experiment sollen folgende Angriffstechniken verwendet werden: \acrfull{ddos}.

Im erstem Experiment wird der Host eine normale Anfrage an den Server schicken. Dieser wird standarmäßig konfiguriert,
also ohne irgendwelche Sicherheitsmechanismen, wie Authentifizierung, Überprüfung der Anzahl von Verbindungen oder Anfragen
nach Zertifikaten.

Der erste Angriff wird mithilfe des Tools \acrfull{nmap}\footnote{\acrshort{nmap} ist eine freie und Open 
Source Anwendung für die Entdeckung und Sicherheitsüberprüfung von Netzwerken \cite{refst:nmap}.} durchgeführt
werden. In diesem Angriff benutzt der Angreifer eine weitere Maschine, um sich selbst zu verbergen und, um den 
Angriff zu verstärken. Der Angreifer schickt gespoofte\footnote{Angreifer verwenden meistens legitimen Adresse 
von anderen Rechner, um die eigene Identität zu verbergen. In diesem Fall ist die eigene Adresse gefälscht 
\cite{refst:IPIO}.} Pakete\footnote{Pakete sind im Netzwerk die Einkapselung von Metainformationen, wie Quell-
und Zieladresse Protokolltyp und Größe die Nutzdaten, wie Text, Videos oder Bilder \cite{refbook:SWIS}.} an 
die zwei \textit{Zombie} und diese schicken sehr viele kleine Pakete in sehr kürzem Abstand an das Server, 
um dessen Kapazität auszulasten, sodass er auf keine Anfragen mehr antworten kann \cite{refip:KSDD}. Im 
folgenden gibt es eine Abbildung zu dieser Angriffstechnik:

\begin{figure}[H]
  \centering{\includegraphics[width=8cm]{Bilder/refip_VDSD.png}}
  \caption{Ein Beispiel von \acrfull{ddos} mit mehreren Leerlaufe-Maschinen
  \\ Quelle: Durcekova et al., 2012}
  \label{fig:VDSD}
\end{figure}
%\cite{refip:VDSD}

\subsubsection{Angriff und Härtungsmaßnahme von Smartcard}
\textbf{Ich würde nur das Beispiel von NFC geben und die anderen nur nach Bedarf hinzufügen, sonst werden wir hier
viele Seite haben}

\textcolor{red}{\textbf{Vllt sollten wir ein Diagramm von diesem Angriff darstellen.}}


\subsection{Beobachtung von Angriffsmöglichkeiten}
\textbf{Hier können wir sagen, dass wir in einem Labor einige Angriffe durchgeführt haben. Wir beschreiben alle Elementen
dieses Labor und was wir von diesem Experimenten erwarten. Auch die Quelle für solche Beobachtung.}

Vor dem Angriff konnte der Host normal mit dem Server kommunizieren, also Anfragen schicken und er hat eine Antwort bekommen.
Während des Angriffes war die Kommunikation mit dem Server entweder sehr langsam oder sogar unterbrochen. In diesem Fall
bekam der Host selten eine Antwort auf seine Anfrage. In einigen Momenten gab es überhaubt keine Antwort. 

Seitens des Servers wurde das Tool Wireshark\footnote{Wireshark ist eine Anwendung für die Analyse von Networkprotoklle.
Es beschreibt ein- und ausgehende Pakete und dessen Bestandteile \cite{refst:wisa}.} verwendet, um die Ein- und Ausgehenden
Pakete zu beobachten und zu analysieren \cite{refart:UBEC}. Unter normalen Umstände kamen die Pakete in einem angemessenen
Zeitabstand. Während des Angriffes bekam der Server viele kleine Pakete ohne nützlichen Inhalt und in sehr kurzem Zeitabstand.
Im folgenden gibt es eine Abbildung, wie Wireshark die Kommunikation aufezeichnet hat:

\begin{figure}[H]
  \centering{\includegraphics[width=8cm]{Bilder/refst_wisa.png}}
  \caption{Ausgabe von Wireshark \\Quelle: Wireshark, 2021}
  \label{fig:refst_wisa}
\end{figure}
%\cite{refst:wisa}

Um den Angriff zu verhindern, schlug \cite{refip:NYRS} vor, den Server erneut zu konfigurieren, indem er nur Anfragen von
registrierten Hosts akzeptiert. Nach dieser Anpassung konnte sich der Angreifer nicht mehr mit dem Server verbinden, da 
er kein registrierter Nutzer war. In der Aufzeichnung on Wireshark wurden nicht angemeldete Pakete direkt verworfen.



\subsection{Literaturrecherche}

Die Literatur bezüglich Netzwerksicherheit, bargeldlose Zahlungsverfahren und Vending Machines, ist in den 
letzten 20 Jahren deutlich umfangreicher geworden. Da diese Begriffe viele und nahezu unendlich Konzepte 
decken, gehen wir hier auf spezifische Aspekte dieser Begriffe ein und zwar auf die Sicherheit von drahtlosen 
Zahlungsmethode und von Smartcards. 

Folgende Quelle trugen zu der Suche nach vertrauenswürdiger Literatur bei:

\begin{itemize}
    \item ScienceDirect
    \item Researchg Gate
    \item IEEE Xplore
    \item Google Scholar.
\end{itemize}

Diese Quellen ermöglichten einen allgemein theoretischen Einblick über das Objekt dieser Untersuchung und
dessen aktuellen Stand. Hier wird hauptsächlich gezielt, die existierende Kenntnis in einer strukturierten
Art und Weise hervorzuheben und zu präsentieren \cite{refbook:SRJR}.
