\section{Forschungsziele}


In diesem Artikel soll ein Konzept für ein Click-and-Buy-Supermarkt direkt neben dem Campingplatz 
entwickelt werden. Solch ein Konzept kann dazu beitragen, dass Campingplätze modernisiert werden 
und noch mehr Touristen angelockt werden. Bevor das Projekt jedoch umgesetzt werden kann, müssen 
noch wichtige Dinge beleuchtet werden. 

Um diesen elektronischen Supermarkt zu entwickeln, ist es wichtig, die gesamte Umgebung solche einer Maschine
zu verstehen: Verfügbarkeit des Netzwerkzugangs, notwendige physische Komponenten, Programmierschnittstellen,
verschiede Arten von Softwaretests und der sichere Umgang mit den Ein- und Ausgaben. Wenn all diese 
Voraussetzungen erfüllt werden, muss am Ende geprüft werden, ob es auch von potenziellen Nutzenden 
akzeptiert wird. 

Wenn es um die Verfügbarkeit des Netzwerkzugangs geht für den Click and Buy Automat, muss zum einen 
geprüft werden, ob die bereits vorhandenen Leitungen ausreichen, um solch ein Projekt umzusetzen. Zum anderen 
sollte die Software für das Click and Buy Systems so konzipiert sein, dass diese eine geringe Ausfallquote
aufweisen, denn dieser soll rund um die Uhr betriebsbereit sein, um die Verfügbarkeit des Systems nicht zu
verletzen \cite{refbook:SWIS}.

Zudem soll das System so entwickelt werden, sodass auch  Digital Non-Natives, die Möglicheit 
haben das System einfach bedienen zu können \cite{refart:QWDN}. Die Kunden sollten also nicht von 
Informationen überladen werden, sondern es sollte einfache Ein- und Ausgaben geben. Da besonderes alteren
Leute sich für solche Urlaubsmöglichkeit entscheiden, spielt die Akzeptanz dieser Leute eine wichtige
Rolle für den Erfolg des Konzepts. Deshalb sollten die Bedürfnissen und Einschränkungen diese Altergruppe
besonders berücksichtig werden, um ihre Vertrauen zu gewinnen \cite{refart:HLAU}.Die Auswahl 
der Tests trägt dazu bei, dass die Zufriedenheit und die Akzeptanz gewährleistet wird, sodass jeder
potenziellen Endnutzer das System bedienen kann \cite{refbook:IASE}.

Außerdem spielt die Sicherheit bei den bargeldlosen Zahlungsvorgängen eine große Rolle und sollte deshalb 
höchste Priorität haben. Verschiedene aktuelle Beispiele von Cyberangriffe zeigen, dass der Umgang mit solchen 
Daten, kritisch zu sehen ist. Es wird oft von Situationen in den Medien berichtet, bei denen Kunden ihr
Geld verloren oder dessen personenbezogenen Daten missbraucht wurden, manchmal sogar von der eigenen Regierung,
nur weil das System nicht ausreichend gegen Angriffe entwicklet wurde. In dieser Hinsicht sollte es bei der 
Entwicklung spezifische und klare Richtlinien berücksichtig werden, so dass der Umgang mit personenbezogenen
Daten sicher bleibt \cite{refart:TRVR}. Um diese Vertraulichkeitsverletzung zu vermeiden, spielt die Konziperiung 
von sicheren bargeldlosen Zahlungsmethoden eine wesentliche Rolle in diesem Artikel. Hier wird hauptsächlich
die folgende Frage behanldet: wie kann sicheres Bezahlen in einem Click and Buy Automat gewährleistet werden?