\section{Forschungsziele}


In diesem Artikel soll ein Konzept für ein Click-and-Buy-Supermarkt direkt neben dem Campingplatz 
entwickelt werden. Solch ein Konzept kann dazu beitragen, dass Campingplätze und die Gegend
modernisiert werden und noch mehr Touristen angelockt werden. Bevor das Projekt jedoch umgesetzt 
werden kann, müssen noch wichtige Dinge beleuchtet werden. 


Um einen elektronischen Supermarkt zu entwickeln, müssen einige Voraussetzungen erfüllt werden.
Der Zugang zum Netzwerk über das Glasfaser sollte immer gewährleistet werden, eine stabile 
Software, die den Qualitätsstandards entspricht, ein sicherer Umgang mit Kundendaten, der sich an 
spezifischen und internationalen Richtlinien \textbf{Fußnote} orientiert, ein benutzerfreundliches System, das sich an 
verschiedenen Kundentypen, wie Alters- und Bildungsgruppe anpasst und letztlich ein kryptographisches
Verfahren \textbf{Fußnote} für das bargeldlose Bezahlen, das die Vertraulichkeit sicherstellt.


Um die Verfügbarkeit des Netzwerkzugangs für den Click-and-Buy-Automat zu gewährleisten, muss zum einen 
geprüft werden, ob die bereits vorhandenen Leitungen ausreichen, um solch ein Projekt umsetzen zu können.
Die Vernetzung soll so aufgebaut sein, dass es auch in remoten Regionen einwandfrei funktioniert. 
Die Software muss zudem so entwickelt werden, sodass diese eine geringe Ausfallquote aufweist, 
denn der Automat soll rund um die Uhr betriebsbereit sein, um das Ziel der Verfügbarkeit des
 Systems nicht zu verletzen \cite{refbook:SWIS}.


Zudem soll das System so entwickelt werden, sodass auch Digital Non-Natives \textbf{Fußnote}, die Möglicheit
\cite{refart:QWDN} haben das System einfach bedienen zu können. Die Kunden sollten also nicht von 
Informationen überladen werden, sondern es sollte einfache Ein- und Ausgaben geben. Da sich besonderes
ältere Menschen für solch eine Urlaubsmöglichkeit entscheiden, spielt es für den Erfolg des Konzeptes 
eine entscheidene Rolle, dass auch sie mit dem Automaten umgehen können. Deshalb sollten die Bedürfnisse
und Einschränkungen dieser Altergruppe besonders berücksichtigt werden, um ihr Vertrauen zu gewinnen
\cite{refart:HLAU} und hauptsächlich gegen Social Engeneering Angriffe \textbf{Fußnote} zu schützen. Die Auswahl der
Tests trägt dazu bei, dass die Zufriedenheit und die Akzeptanz gewährleistet wird, sodass jeder 
potenziellen Endnutzer das System bedienen kann \cite{refbook:IASE}.

Außerdem spielt die Sicherheit bei den bargeldlosen Zahlungsvorgängen eine große Rolle und sollte
deshalb höchste Priorität haben. Verschiedene aktuelle Beispiele von Cyberangriffe zeigen, dass der 
Umgang mit solchen Daten, kritisch zu sehen ist. Es wird oft von Situationen in den Medien berichtet,
bei denen Kunden ihr Geld verloren haben oder dessen personenbezogenen Daten missbraucht wurden. 
In seltenen Fällen sogar von der eigenen Regierung, weil das System nicht ausreichend gegen Angriffe 
geschützt wurde. In dieser Hinsicht sollten bei der Entwicklung spezifische und klare Richtlinien 
berücksichtig werden, sodass der sichere Umgang mit personenbezogenen Daten gewährleistet ist 
\cite{refart:TRVR}. Um diese Vertraulichkeitsverletzung zu vermeiden, spielt die Konziperiung von 
sicheren bargeldlosen Zahlungsmethoden eine wesentliche Rolle in diesem Artikel. 

Da das gesamte Thema sehr umfangreich ist, soll hier hauptsächlich die folgende Frage behanldet werden: 
Wie kann sicheres bargeldloses Bezahlen in einem Click-and-Buy-Automat gewährleistet werden?

\textbf{Diese Frage geht am Ende der Einführung. Formulierung}
