\section{Forschungsziele}


In diesem Artikel soll ein Konzept für ein Click-and-Buy-Supermarkt direkt neben dem Campingplatz 
entwickelt werden. Solche ein Konzept kan dazu beitragen, dass Campingplätze modernisiert werden 
und mehr Urlauber bekommen. Bevor eine Firma beauftragt wird, die Anforderungen auszuführen, 
müssen einige Thema noch bearbeitet werden. Diese beziehen sich auf folgende Punkte: 

\begin{enumerate}
    \item existierende Infrastruktur;
    \item Interesse von potenziellen Kunden und Firmen für solche ein Projekt;
\end{enumerate}


\subsection{Ziele aus dem Informatikbereich}

Technisch gesehen ist es möglich die Komplexität solche Entwicklung in verschiedenen Zielen unterteilen:

\begin{enumerate}
    \item Gewährleistung des Netzwerkzugang in dem Ort;
    \item Wartungen und Weiterentwicklung der Geräte;
    \item Bezahlungsmöglichkeit;
    \item Sicherumgang mit eigegebenen Daten;
    \item Programmierung eine benutzerfreundliche Schnittstelle;
    \item Akzeptanz der gesamten Konzept von den potenziellen Kunden.
\end{enumerate}

Zusätzlich hat der Sicherheit bei dem Zahlungsverfahren höchste Priorität, denn es ist besonders kritsich, 
wenn bei einem Bezahlvorgang Daten abgefangen werden. In dieser Hinsicht konzentriert sich diese Arbeit
auf die Entwicklung ein Bezahlunsverfahren, das gewährleisten kann, dass Bezahlung sicher werden. 

%
%
%Zudem spielt die Usability des Click and Buy Systems eine große Rolle, denn egal ob alt oder jung, 
%jeder sollte diesen Automaten einfach bedienen können. Die Kunden sollten also nicht von Inforamtionen überladen werden, 
%sondern es sollte einfache Ein- und Ausgaben geben.
%
%Eine andere und wichtige Herausforderung bezieht sich auf den Netzwerkzugang in Orten, die für die schwierige 
%Erreichbarkeit bekannt sind. In diesem Fall sollten der Aufbau und die Einstellungen der Maschine gewährleisten,
%dass der Zugang rund um die Uhr funktioniert.
%

%
%Dazu werden wir eine Gruppe von Menschen in verschiedenen Altersgruppen 
%den Automaten im Vorfeld testen lassen, sodass wir eine erste Rückmeldung über die Bedienbarkeit des Automaten bekommen.
%Natürlich müssen auch noch weitere Tests durchgeführt werden, diese werden in den weiteren Kapiteln behandelt.
%
%
%Solche eine Annehmlichkeit setzten viele Vorbereitung im voraus, sodass es überall einwendafrei funktionieren kann,
%besonders wo die Erreichbarkeit schwierig ist. 
%
%
\subsection{Ziele aus dem Tourismusbereich}


Aus dem touristischen Bereich ist es notwendig zu verstehen, wer zu der Zielgruppe dieser Urlaubsmodalität gehört
und was für Artikeln für den Verkauf wichtig sind.


Zu wissen was für Produkten in solchen Urlaubsmöglichkeit notwendig sind, macht ein großer Unterschied, ob sich die
Idee von diesem Automat akzeptiert ist. In diesem Fall wäre eine Umfrage an Urlauber notwendig, um Daten über 
potenzielle Nutzenden zu sammeln. Aus dieser Umfrage konnte es auch Informationen hervorgehoben werden, die sich auf
mögliche Produkten beziehen, die bei solche Urlaubsziele mitgenommen werden.

---------------------------------------------------------------------------------------------------------------------